\documentclass[]{article}
\usepackage{lmodern}
\usepackage{amssymb,amsmath}
\usepackage{ifxetex,ifluatex}
\usepackage{fixltx2e} % provides \textsubscript
\ifnum 0\ifxetex 1\fi\ifluatex 1\fi=0 % if pdftex
  \usepackage[T1]{fontenc}
  \usepackage[utf8]{inputenc}
\else % if luatex or xelatex
  \ifxetex
    \usepackage{mathspec}
  \else
    \usepackage{fontspec}
  \fi
  \defaultfontfeatures{Ligatures=TeX,Scale=MatchLowercase}
\fi
% use upquote if available, for straight quotes in verbatim environments
\IfFileExists{upquote.sty}{\usepackage{upquote}}{}
% use microtype if available
\IfFileExists{microtype.sty}{%
\usepackage{microtype}
\UseMicrotypeSet[protrusion]{basicmath} % disable protrusion for tt fonts
}{}
\usepackage[margin=1in]{geometry}
\usepackage{hyperref}
\hypersetup{unicode=true,
            pdftitle={Climbing Insights},
            pdfauthor={Ian McDonald},
            pdfborder={0 0 0},
            breaklinks=true}
\urlstyle{same}  % don't use monospace font for urls
\usepackage{color}
\usepackage{fancyvrb}
\newcommand{\VerbBar}{|}
\newcommand{\VERB}{\Verb[commandchars=\\\{\}]}
\DefineVerbatimEnvironment{Highlighting}{Verbatim}{commandchars=\\\{\}}
% Add ',fontsize=\small' for more characters per line
\usepackage{framed}
\definecolor{shadecolor}{RGB}{248,248,248}
\newenvironment{Shaded}{\begin{snugshade}}{\end{snugshade}}
\newcommand{\AlertTok}[1]{\textcolor[rgb]{0.94,0.16,0.16}{#1}}
\newcommand{\AnnotationTok}[1]{\textcolor[rgb]{0.56,0.35,0.01}{\textbf{\textit{#1}}}}
\newcommand{\AttributeTok}[1]{\textcolor[rgb]{0.77,0.63,0.00}{#1}}
\newcommand{\BaseNTok}[1]{\textcolor[rgb]{0.00,0.00,0.81}{#1}}
\newcommand{\BuiltInTok}[1]{#1}
\newcommand{\CharTok}[1]{\textcolor[rgb]{0.31,0.60,0.02}{#1}}
\newcommand{\CommentTok}[1]{\textcolor[rgb]{0.56,0.35,0.01}{\textit{#1}}}
\newcommand{\CommentVarTok}[1]{\textcolor[rgb]{0.56,0.35,0.01}{\textbf{\textit{#1}}}}
\newcommand{\ConstantTok}[1]{\textcolor[rgb]{0.00,0.00,0.00}{#1}}
\newcommand{\ControlFlowTok}[1]{\textcolor[rgb]{0.13,0.29,0.53}{\textbf{#1}}}
\newcommand{\DataTypeTok}[1]{\textcolor[rgb]{0.13,0.29,0.53}{#1}}
\newcommand{\DecValTok}[1]{\textcolor[rgb]{0.00,0.00,0.81}{#1}}
\newcommand{\DocumentationTok}[1]{\textcolor[rgb]{0.56,0.35,0.01}{\textbf{\textit{#1}}}}
\newcommand{\ErrorTok}[1]{\textcolor[rgb]{0.64,0.00,0.00}{\textbf{#1}}}
\newcommand{\ExtensionTok}[1]{#1}
\newcommand{\FloatTok}[1]{\textcolor[rgb]{0.00,0.00,0.81}{#1}}
\newcommand{\FunctionTok}[1]{\textcolor[rgb]{0.00,0.00,0.00}{#1}}
\newcommand{\ImportTok}[1]{#1}
\newcommand{\InformationTok}[1]{\textcolor[rgb]{0.56,0.35,0.01}{\textbf{\textit{#1}}}}
\newcommand{\KeywordTok}[1]{\textcolor[rgb]{0.13,0.29,0.53}{\textbf{#1}}}
\newcommand{\NormalTok}[1]{#1}
\newcommand{\OperatorTok}[1]{\textcolor[rgb]{0.81,0.36,0.00}{\textbf{#1}}}
\newcommand{\OtherTok}[1]{\textcolor[rgb]{0.56,0.35,0.01}{#1}}
\newcommand{\PreprocessorTok}[1]{\textcolor[rgb]{0.56,0.35,0.01}{\textit{#1}}}
\newcommand{\RegionMarkerTok}[1]{#1}
\newcommand{\SpecialCharTok}[1]{\textcolor[rgb]{0.00,0.00,0.00}{#1}}
\newcommand{\SpecialStringTok}[1]{\textcolor[rgb]{0.31,0.60,0.02}{#1}}
\newcommand{\StringTok}[1]{\textcolor[rgb]{0.31,0.60,0.02}{#1}}
\newcommand{\VariableTok}[1]{\textcolor[rgb]{0.00,0.00,0.00}{#1}}
\newcommand{\VerbatimStringTok}[1]{\textcolor[rgb]{0.31,0.60,0.02}{#1}}
\newcommand{\WarningTok}[1]{\textcolor[rgb]{0.56,0.35,0.01}{\textbf{\textit{#1}}}}
\usepackage{graphicx,grffile}
\makeatletter
\def\maxwidth{\ifdim\Gin@nat@width>\linewidth\linewidth\else\Gin@nat@width\fi}
\def\maxheight{\ifdim\Gin@nat@height>\textheight\textheight\else\Gin@nat@height\fi}
\makeatother
% Scale images if necessary, so that they will not overflow the page
% margins by default, and it is still possible to overwrite the defaults
% using explicit options in \includegraphics[width, height, ...]{}
\setkeys{Gin}{width=\maxwidth,height=\maxheight,keepaspectratio}
\IfFileExists{parskip.sty}{%
\usepackage{parskip}
}{% else
\setlength{\parindent}{0pt}
\setlength{\parskip}{6pt plus 2pt minus 1pt}
}
\setlength{\emergencystretch}{3em}  % prevent overfull lines
\providecommand{\tightlist}{%
  \setlength{\itemsep}{0pt}\setlength{\parskip}{0pt}}
\setcounter{secnumdepth}{0}
% Redefines (sub)paragraphs to behave more like sections
\ifx\paragraph\undefined\else
\let\oldparagraph\paragraph
\renewcommand{\paragraph}[1]{\oldparagraph{#1}\mbox{}}
\fi
\ifx\subparagraph\undefined\else
\let\oldsubparagraph\subparagraph
\renewcommand{\subparagraph}[1]{\oldsubparagraph{#1}\mbox{}}
\fi

%%% Use protect on footnotes to avoid problems with footnotes in titles
\let\rmarkdownfootnote\footnote%
\def\footnote{\protect\rmarkdownfootnote}

%%% Change title format to be more compact
\usepackage{titling}

% Create subtitle command for use in maketitle
\providecommand{\subtitle}[1]{
  \posttitle{
    \begin{center}\large#1\end{center}
    }
}

\setlength{\droptitle}{-2em}

  \title{Climbing Insights}
    \pretitle{\vspace{\droptitle}\centering\huge}
  \posttitle{\par}
    \author{Ian McDonald}
    \preauthor{\centering\large\emph}
  \postauthor{\par}
    \date{}
    \predate{}\postdate{}
  

\begin{document}
\maketitle

Load libraries that will be needed in this file.

\begin{Shaded}
\begin{Highlighting}[]
\KeywordTok{library}\NormalTok{(}\StringTok{"knitr"}\NormalTok{)}
\KeywordTok{library}\NormalTok{(}\StringTok{"caret"}\NormalTok{)}
\end{Highlighting}
\end{Shaded}

\begin{verbatim}
## Loading required package: lattice
\end{verbatim}

\begin{verbatim}
## Loading required package: ggplot2
\end{verbatim}

Add perviously done work from the part2.Rmd file

\begin{Shaded}
\begin{Highlighting}[]
\KeywordTok{purl}\NormalTok{(}\StringTok{"part2.Rmd"}\NormalTok{, }\DataTypeTok{output =} \StringTok{"part2.r"}\NormalTok{)}
\end{Highlighting}
\end{Shaded}

\begin{verbatim}
## 
## 
## processing file: part2.Rmd
\end{verbatim}

\begin{verbatim}
## 
  |                                                                            
  |                                                                      |   0%
  |                                                                            
  |....                                                                  |   6%
  |                                                                            
  |........                                                              |  11%
  |                                                                            
  |............                                                          |  17%
  |                                                                            
  |................                                                      |  22%
  |                                                                            
  |...................                                                   |  28%
  |                                                                            
  |.......................                                               |  33%
  |                                                                            
  |...........................                                           |  39%
  |                                                                            
  |...............................                                       |  44%
  |                                                                            
  |...................................                                   |  50%
  |                                                                            
  |.......................................                               |  56%
  |                                                                            
  |...........................................                           |  61%
  |                                                                            
  |...............................................                       |  67%
  |                                                                            
  |...................................................                   |  72%
  |                                                                            
  |......................................................                |  78%
  |                                                                            
  |..........................................................            |  83%
  |                                                                            
  |..............................................................        |  89%
  |                                                                            
  |..................................................................    |  94%
  |                                                                            
  |......................................................................| 100%
\end{verbatim}

\begin{verbatim}
## output file: part2.r
\end{verbatim}

\begin{verbatim}
## [1] "part2.r"
\end{verbatim}

\begin{Shaded}
\begin{Highlighting}[]
\KeywordTok{source}\NormalTok{(}\StringTok{"part2.r"}\NormalTok{)}
\end{Highlighting}
\end{Shaded}

\begin{verbatim}
## 
## 
## processing file: part1.Rmd
\end{verbatim}

\begin{verbatim}
## 
  |                                                                            
  |                                                                      |   0%
  |                                                                            
  |..                                                                    |   3%
  |                                                                            
  |.....                                                                 |   7%
  |                                                                            
  |.......                                                               |  10%
  |                                                                            
  |..........                                                            |  14%
  |                                                                            
  |............                                                          |  17%
  |                                                                            
  |..............                                                        |  21%
  |                                                                            
  |.................                                                     |  24%
  |                                                                            
  |...................                                                   |  28%
  |                                                                            
  |......................                                                |  31%
  |                                                                            
  |........................                                              |  34%
  |                                                                            
  |...........................                                           |  38%
  |                                                                            
  |.............................                                         |  41%
  |                                                                            
  |...............................                                       |  45%
  |                                                                            
  |..................................                                    |  48%
  |                                                                            
  |....................................                                  |  52%
  |                                                                            
  |.......................................                               |  55%
  |                                                                            
  |.........................................                             |  59%
  |                                                                            
  |...........................................                           |  62%
  |                                                                            
  |..............................................                        |  66%
  |                                                                            
  |................................................                      |  69%
  |                                                                            
  |...................................................                   |  72%
  |                                                                            
  |.....................................................                 |  76%
  |                                                                            
  |........................................................              |  79%
  |                                                                            
  |..........................................................            |  83%
  |                                                                            
  |............................................................          |  86%
  |                                                                            
  |...............................................................       |  90%
  |                                                                            
  |.................................................................     |  93%
  |                                                                            
  |....................................................................  |  97%
  |                                                                            
  |......................................................................| 100%
\end{verbatim}

\begin{verbatim}
## output file: part1.r
\end{verbatim}

\begin{verbatim}
## -- Attaching packages ------------------------------------------------------------------------------ tidyverse 1.2.1 --
\end{verbatim}

\begin{verbatim}
## v tibble  2.1.3     v purrr   0.3.2
## v tidyr   1.0.0     v dplyr   0.8.3
## v readr   1.3.1     v stringr 1.4.0
## v tibble  2.1.3     v forcats 0.4.0
\end{verbatim}

\begin{verbatim}
## -- Conflicts --------------------------------------------------------------------------------- tidyverse_conflicts() --
## x dplyr::filter() masks stats::filter()
## x dplyr::lag()    masks stats::lag()
## x purrr::lift()   masks caret::lift()
\end{verbatim}

After seeing that height wasnt very significant in the last sections I
decided to remove it from my model before trying to validate.

\begin{Shaded}
\begin{Highlighting}[]
\NormalTok{new_grade_model <-}\StringTok{ }\KeywordTok{lm}\NormalTok{(train_data, }\DataTypeTok{formula =}\NormalTok{ grade_id }\OperatorTok{~}\StringTok{ }\NormalTok{weight }\OperatorTok{+}\StringTok{ }\NormalTok{started }\OperatorTok{+}\StringTok{ }\NormalTok{sex)}
\KeywordTok{summary}\NormalTok{(new_grade_model)}
\end{Highlighting}
\end{Shaded}

\begin{verbatim}
## 
## Call:
## lm(formula = grade_id ~ weight + started + sex, data = train_data)
## 
## Residuals:
##     Min      1Q  Median      3Q     Max 
## -54.942  -5.257   1.005   6.208  32.736 
## 
## Coefficients:
##               Estimate Std. Error t value Pr(>|t|)    
## (Intercept) 844.031999  12.433801   67.88   <2e-16 ***
## weight       -0.224215   0.005257  -42.65   <2e-16 ***
## started      -0.385824   0.006186  -62.37   <2e-16 ***
## sex          -7.778821   0.169404  -45.92   <2e-16 ***
## ---
## Signif. codes:  0 '***' 0.001 '**' 0.01 '*' 0.05 '.' 0.1 ' ' 1
## 
## Residual standard error: 8.702 on 34064 degrees of freedom
## Multiple R-squared:  0.1591, Adjusted R-squared:  0.159 
## F-statistic:  2149 on 3 and 34064 DF,  p-value: < 2.2e-16
\end{verbatim}

Now we want to validate our model to test the accuracy of it. This test
how accurate our model is using the predicted values. Before doing this
we need to clean the data back to whole numbers since climbs are based
on the numbers and cant have decimals. Then we use the RMSE functions to
calculate our value. lowest test sample RMSE is the preferred model.

\begin{Shaded}
\begin{Highlighting}[]
\NormalTok{grade_predictions <-}\StringTok{ }\NormalTok{new_grade_model }\OperatorTok\StringTok{ }\KeywordTok{predict}\NormalTok{(test_data)}
\NormalTok{grade_predictions <-}\StringTok{ }\KeywordTok{as.data.frame}\NormalTok{(grade_predictions)}
\KeywordTok{colnames}\NormalTok{(grade_predictions)[}\KeywordTok{colnames}\NormalTok{(grade_predictions) }\OperatorTok{==}\StringTok{ "grade_predictions"}\NormalTok{] <-}\StringTok{ "grade_id"}
\NormalTok{grade_predictions[}\StringTok{"grade_id"}\NormalTok{] <-}\StringTok{ }\KeywordTok{floor}\NormalTok{(grade_predictions[}\StringTok{"grade_id"}\NormalTok{])}
\KeywordTok{RMSE}\NormalTok{(grade_predictions, test_data}\OperatorTok{$}\NormalTok{grade_id)}\OperatorTok{/}\KeywordTok{mean}\NormalTok{(test_data}\OperatorTok{$}\NormalTok{grade_id)}
\end{Highlighting}
\end{Shaded}

\begin{verbatim}
## [1] 0.1591709
\end{verbatim}

After exploring the results of my model it shows us that their are
several factors that are very influential in predicting the highest
grade someone has climbed. Some of the things that weren's very
suprising were things like weight, sex, and year started climbing. Below
is a visualization of some of our inital data that lead me to these
conclusions. Additionally I was quite suprised to find that there was
not a statistically significant relation between height and the highest
grade someone has climbed. If we visualzed some of the other factors
such as weight and year started I would expect to see similar graphs in
which the visualizations help us see what the model shows.

\begin{Shaded}
\begin{Highlighting}[]
\NormalTok{zero_df <-}\StringTok{ }\NormalTok{hardestWithYear[}\KeywordTok{which}\NormalTok{(hardestWithYear[}\StringTok{"sex"}\NormalTok{]}\OperatorTok{==}\DecValTok{0}\NormalTok{),]}
\NormalTok{one_df <-}\StringTok{ }\NormalTok{hardestWithYear[}\KeywordTok{which}\NormalTok{(hardestWithYear[}\StringTok{"sex"}\NormalTok{]}\OperatorTok{==}\DecValTok{1}\NormalTok{),]}
\KeywordTok{mean}\NormalTok{(zero_df}\OperatorTok{$}\NormalTok{grade_id)}
\end{Highlighting}
\end{Shaded}

\begin{verbatim}
## [1] 54.85335
\end{verbatim}

\begin{Shaded}
\begin{Highlighting}[]
\KeywordTok{mean}\NormalTok{(one_df}\OperatorTok{$}\NormalTok{grade_id)}
\end{Highlighting}
\end{Shaded}

\begin{verbatim}
## [1] 49.9177
\end{verbatim}

\begin{Shaded}
\begin{Highlighting}[]
\NormalTok{avg_df <-}\StringTok{ }\KeywordTok{as.data.frame}\NormalTok{(}\KeywordTok{c}\NormalTok{(}\KeywordTok{mean}\NormalTok{(zero_df}\OperatorTok{$}\NormalTok{grade_id), }\KeywordTok{mean}\NormalTok{(one_df}\OperatorTok{$}\NormalTok{grade_id)))}
\KeywordTok{colnames}\NormalTok{(avg_df)[}\KeywordTok{colnames}\NormalTok{(avg_df) }\OperatorTok{==}\StringTok{ "c(mean(zero_df$grade_id), mean(one_df$grade_id))"}\NormalTok{] <-}\StringTok{ "avg_grade"}
\NormalTok{gender <-}\StringTok{ }\KeywordTok{c}\NormalTok{(}\StringTok{"male"}\NormalTok{, }\StringTok{"female"}\NormalTok{)}
\NormalTok{avg_df <-}\StringTok{ }\KeywordTok{cbind}\NormalTok{(avg_df, gender)}
\NormalTok{f <-}\StringTok{ }\KeywordTok{ggplot}\NormalTok{(avg_df, }\KeywordTok{aes}\NormalTok{(gender, avg_grade)) }\OperatorTok{+}\KeywordTok{geom_col}\NormalTok{()}
\NormalTok{f}
\end{Highlighting}
\end{Shaded}

\includegraphics{part3_files/figure-latex/unnamed-chunk-5-1.pdf}

Finally when looking at operationalizing the project we must consider
what it is useful for. This sort of predictor could be useful to help
climbers know about where they relate to other climbers. If they are
climbing a much lower grade than someone with similar stats then they
might be able to infer a weakness they have and need to work on to
progress much quicker. Additionally we could expand this model to
predict per route. If we did something like that then climbers could use
it to determine the odds they have at climbing a certain route. To
further operationalize the model we would have to continue adding to its
data. To do this we could continue to scrape 8a.nu for new entries as
climbing is a sport that is ever evolving. Finally in its current state
there are some issues that could arrise of the model socially. Climbing
is a sport that has a rich history of competitors having eating
disorders. It is little suprised that the model showed weight as a
massive predictor. Competition climbers are usually very lean athletes
who go on extremem diets to prepare for competitions. At any level of
climbing the less weight you have to take up the wall the easier it will
be. Having data like this know could lead to some people wanting to lose
weight to climb harder and they might go about it in extreme ways. This
could further increase the severity of an already problematic attitude
towards climbing. Additionally this data is probably somewhat biased
towards more hardcore climbers so they numbers might not be representive
of all climbers. This is due to the fact that 8a.nu is a website for
more serious climbers to log climbs and is not used by many newer
climbers.


\end{document}
