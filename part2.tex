\documentclass[]{article}
\usepackage{lmodern}
\usepackage{amssymb,amsmath}
\usepackage{ifxetex,ifluatex}
\usepackage{fixltx2e} % provides \textsubscript
\ifnum 0\ifxetex 1\fi\ifluatex 1\fi=0 % if pdftex
  \usepackage[T1]{fontenc}
  \usepackage[utf8]{inputenc}
\else % if luatex or xelatex
  \ifxetex
    \usepackage{mathspec}
  \else
    \usepackage{fontspec}
  \fi
  \defaultfontfeatures{Ligatures=TeX,Scale=MatchLowercase}
\fi
% use upquote if available, for straight quotes in verbatim environments
\IfFileExists{upquote.sty}{\usepackage{upquote}}{}
% use microtype if available
\IfFileExists{microtype.sty}{%
\usepackage{microtype}
\UseMicrotypeSet[protrusion]{basicmath} % disable protrusion for tt fonts
}{}
\usepackage[margin=1in]{geometry}
\usepackage{hyperref}
\hypersetup{unicode=true,
            pdftitle={Climbing Insights},
            pdfauthor={Ian McDonald},
            pdfborder={0 0 0},
            breaklinks=true}
\urlstyle{same}  % don't use monospace font for urls
\usepackage{color}
\usepackage{fancyvrb}
\newcommand{\VerbBar}{|}
\newcommand{\VERB}{\Verb[commandchars=\\\{\}]}
\DefineVerbatimEnvironment{Highlighting}{Verbatim}{commandchars=\\\{\}}
% Add ',fontsize=\small' for more characters per line
\usepackage{framed}
\definecolor{shadecolor}{RGB}{248,248,248}
\newenvironment{Shaded}{\begin{snugshade}}{\end{snugshade}}
\newcommand{\AlertTok}[1]{\textcolor[rgb]{0.94,0.16,0.16}{#1}}
\newcommand{\AnnotationTok}[1]{\textcolor[rgb]{0.56,0.35,0.01}{\textbf{\textit{#1}}}}
\newcommand{\AttributeTok}[1]{\textcolor[rgb]{0.77,0.63,0.00}{#1}}
\newcommand{\BaseNTok}[1]{\textcolor[rgb]{0.00,0.00,0.81}{#1}}
\newcommand{\BuiltInTok}[1]{#1}
\newcommand{\CharTok}[1]{\textcolor[rgb]{0.31,0.60,0.02}{#1}}
\newcommand{\CommentTok}[1]{\textcolor[rgb]{0.56,0.35,0.01}{\textit{#1}}}
\newcommand{\CommentVarTok}[1]{\textcolor[rgb]{0.56,0.35,0.01}{\textbf{\textit{#1}}}}
\newcommand{\ConstantTok}[1]{\textcolor[rgb]{0.00,0.00,0.00}{#1}}
\newcommand{\ControlFlowTok}[1]{\textcolor[rgb]{0.13,0.29,0.53}{\textbf{#1}}}
\newcommand{\DataTypeTok}[1]{\textcolor[rgb]{0.13,0.29,0.53}{#1}}
\newcommand{\DecValTok}[1]{\textcolor[rgb]{0.00,0.00,0.81}{#1}}
\newcommand{\DocumentationTok}[1]{\textcolor[rgb]{0.56,0.35,0.01}{\textbf{\textit{#1}}}}
\newcommand{\ErrorTok}[1]{\textcolor[rgb]{0.64,0.00,0.00}{\textbf{#1}}}
\newcommand{\ExtensionTok}[1]{#1}
\newcommand{\FloatTok}[1]{\textcolor[rgb]{0.00,0.00,0.81}{#1}}
\newcommand{\FunctionTok}[1]{\textcolor[rgb]{0.00,0.00,0.00}{#1}}
\newcommand{\ImportTok}[1]{#1}
\newcommand{\InformationTok}[1]{\textcolor[rgb]{0.56,0.35,0.01}{\textbf{\textit{#1}}}}
\newcommand{\KeywordTok}[1]{\textcolor[rgb]{0.13,0.29,0.53}{\textbf{#1}}}
\newcommand{\NormalTok}[1]{#1}
\newcommand{\OperatorTok}[1]{\textcolor[rgb]{0.81,0.36,0.00}{\textbf{#1}}}
\newcommand{\OtherTok}[1]{\textcolor[rgb]{0.56,0.35,0.01}{#1}}
\newcommand{\PreprocessorTok}[1]{\textcolor[rgb]{0.56,0.35,0.01}{\textit{#1}}}
\newcommand{\RegionMarkerTok}[1]{#1}
\newcommand{\SpecialCharTok}[1]{\textcolor[rgb]{0.00,0.00,0.00}{#1}}
\newcommand{\SpecialStringTok}[1]{\textcolor[rgb]{0.31,0.60,0.02}{#1}}
\newcommand{\StringTok}[1]{\textcolor[rgb]{0.31,0.60,0.02}{#1}}
\newcommand{\VariableTok}[1]{\textcolor[rgb]{0.00,0.00,0.00}{#1}}
\newcommand{\VerbatimStringTok}[1]{\textcolor[rgb]{0.31,0.60,0.02}{#1}}
\newcommand{\WarningTok}[1]{\textcolor[rgb]{0.56,0.35,0.01}{\textbf{\textit{#1}}}}
\usepackage{graphicx,grffile}
\makeatletter
\def\maxwidth{\ifdim\Gin@nat@width>\linewidth\linewidth\else\Gin@nat@width\fi}
\def\maxheight{\ifdim\Gin@nat@height>\textheight\textheight\else\Gin@nat@height\fi}
\makeatother
% Scale images if necessary, so that they will not overflow the page
% margins by default, and it is still possible to overwrite the defaults
% using explicit options in \includegraphics[width, height, ...]{}
\setkeys{Gin}{width=\maxwidth,height=\maxheight,keepaspectratio}
\IfFileExists{parskip.sty}{%
\usepackage{parskip}
}{% else
\setlength{\parindent}{0pt}
\setlength{\parskip}{6pt plus 2pt minus 1pt}
}
\setlength{\emergencystretch}{3em}  % prevent overfull lines
\providecommand{\tightlist}{%
  \setlength{\itemsep}{0pt}\setlength{\parskip}{0pt}}
\setcounter{secnumdepth}{0}
% Redefines (sub)paragraphs to behave more like sections
\ifx\paragraph\undefined\else
\let\oldparagraph\paragraph
\renewcommand{\paragraph}[1]{\oldparagraph{#1}\mbox{}}
\fi
\ifx\subparagraph\undefined\else
\let\oldsubparagraph\subparagraph
\renewcommand{\subparagraph}[1]{\oldsubparagraph{#1}\mbox{}}
\fi

%%% Use protect on footnotes to avoid problems with footnotes in titles
\let\rmarkdownfootnote\footnote%
\def\footnote{\protect\rmarkdownfootnote}

%%% Change title format to be more compact
\usepackage{titling}

% Create subtitle command for use in maketitle
\providecommand{\subtitle}[1]{
  \posttitle{
    \begin{center}\large#1\end{center}
    }
}

\setlength{\droptitle}{-2em}

  \title{Climbing Insights}
    \pretitle{\vspace{\droptitle}\centering\huge}
  \posttitle{\par}
    \author{Ian McDonald}
    \preauthor{\centering\large\emph}
  \postauthor{\par}
    \date{}
    \predate{}\postdate{}
  

\begin{document}
\maketitle

Load libraries that will be needed in this file.

\begin{Shaded}
\begin{Highlighting}[]
\KeywordTok{library}\NormalTok{(}\StringTok{"knitr"}\NormalTok{)}
\KeywordTok{library}\NormalTok{(}\StringTok{"caret"}\NormalTok{)}
\end{Highlighting}
\end{Shaded}

\begin{verbatim}
## Loading required package: lattice
\end{verbatim}

\begin{verbatim}
## Loading required package: ggplot2
\end{verbatim}

Add perviously done work from the musicInsights.Rmd file

\begin{Shaded}
\begin{Highlighting}[]
\KeywordTok{purl}\NormalTok{(}\StringTok{"index.Rmd"}\NormalTok{, }\DataTypeTok{output =} \StringTok{"part1.r"}\NormalTok{)}
\end{Highlighting}
\end{Shaded}

\begin{verbatim}
## 
## 
## processing file: index.Rmd
\end{verbatim}

\begin{verbatim}
## 
  |                                                                            
  |                                                                      |   0%
  |                                                                            
  |..                                                                    |   4%
  |                                                                            
  |.....                                                                 |   7%
  |                                                                            
  |........                                                              |  11%
  |                                                                            
  |..........                                                            |  14%
  |                                                                            
  |............                                                          |  18%
  |                                                                            
  |...............                                                       |  21%
  |                                                                            
  |..................                                                    |  25%
  |                                                                            
  |....................                                                  |  29%
  |                                                                            
  |......................                                                |  32%
  |                                                                            
  |.........................                                             |  36%
  |                                                                            
  |............................                                          |  39%
  |                                                                            
  |..............................                                        |  43%
  |                                                                            
  |................................                                      |  46%
  |                                                                            
  |...................................                                   |  50%
  |                                                                            
  |......................................                                |  54%
  |                                                                            
  |........................................                              |  57%
  |                                                                            
  |..........................................                            |  61%
  |                                                                            
  |.............................................                         |  64%
  |                                                                            
  |................................................                      |  68%
  |                                                                            
  |..................................................                    |  71%
  |                                                                            
  |....................................................                  |  75%
  |                                                                            
  |.......................................................               |  79%
  |                                                                            
  |..........................................................            |  82%
  |                                                                            
  |............................................................          |  86%
  |                                                                            
  |..............................................................        |  89%
  |                                                                            
  |.................................................................     |  93%
  |                                                                            
  |....................................................................  |  96%
  |                                                                            
  |......................................................................| 100%
\end{verbatim}

\begin{verbatim}
## output file: part1.r
\end{verbatim}

\begin{verbatim}
## [1] "part1.r"
\end{verbatim}

\begin{Shaded}
\begin{Highlighting}[]
\KeywordTok{source}\NormalTok{(}\StringTok{"part1.r"}\NormalTok{)}
\end{Highlighting}
\end{Shaded}

\begin{verbatim}
## -- Attaching packages ------------------------------------------------------------------------------ tidyverse 1.2.1 --
\end{verbatim}

\begin{verbatim}
## v tibble  2.1.3     v purrr   0.3.2
## v tidyr   1.0.0     v dplyr   0.8.3
## v readr   1.3.1     v stringr 1.4.0
## v tibble  2.1.3     v forcats 0.4.0
\end{verbatim}

\begin{verbatim}
## -- Conflicts --------------------------------------------------------------------------------- tidyverse_conflicts() --
## x dplyr::filter() masks stats::filter()
## x dplyr::lag()    masks stats::lag()
## x purrr::lift()   masks caret::lift()
\end{verbatim}

One of the things I would like to model is information about individual
users and their hardest climbs. In climbing there are many factors that
climbers may think influence the hardest they can climb. Often some of
the factors that are thought of are physical factors such as height,
weight and sex. Some of the other factors that come into play are more
specific to the climb and we cant analyze them. However one of these
factors we can examine is the time since a climber started climbing.
Using these factors we can hopefully make a model to determine a persons
hardest climb accurately. My goal is to find out what things can predict
a users hardest climb.

First we will be creating a dataset to train a model and compare
results. This code will create a list of the column names we wish to
keep from our large data set that holds our info about climbers and
their hardest climbs. After we have this list we keep only those columns
and create a new dataframe. Finally we omit any rows in the data that
hold any columns containing an NA value.

\begin{Shaded}
\begin{Highlighting}[]
\NormalTok{myvars <-}\StringTok{ }\KeywordTok{c}\NormalTok{(}\StringTok{"grade_id"}\NormalTok{, }\StringTok{"height"}\NormalTok{, }\StringTok{"weight"}\NormalTok{, }\StringTok{"started"}\NormalTok{, }\StringTok{"sex"}\NormalTok{)}
\NormalTok{grade_model_data <-}\StringTok{ }\NormalTok{hardestWithYear[myvars]}
\NormalTok{grade_model_data <-}\StringTok{ }\KeywordTok{na.omit}\NormalTok{(grade_model_data)}
\end{Highlighting}
\end{Shaded}

splitting the previosly made data into two sets. One to train a model
and the other to determine how the model could do in prediciting highest
grade using the remaining data. We are using a validation set here to
evaluate the models performance Additionally we set the seed so the
results are reproduceable.

\begin{Shaded}
\begin{Highlighting}[]
\KeywordTok{set.seed}\NormalTok{(}\DecValTok{926}\NormalTok{)}
\NormalTok{sample <-}\StringTok{ }\KeywordTok{sample.int}\NormalTok{(}\DataTypeTok{n =} \KeywordTok{nrow}\NormalTok{(grade_model_data), }
                     \DataTypeTok{size =} \KeywordTok{floor}\NormalTok{(.}\DecValTok{90}\OperatorTok{*}\KeywordTok{nrow}\NormalTok{(grade_model_data)), }\CommentTok{# Selecting 90% of data}
                     \DataTypeTok{replace =}\NormalTok{ F)}

\NormalTok{train_data <-}\StringTok{ }\NormalTok{grade_model_data[sample, ]}
\NormalTok{test_data  <-}\StringTok{ }\NormalTok{grade_model_data[}\OperatorTok{-}\NormalTok{sample, ]}
\end{Highlighting}
\end{Shaded}

Train a model using our previously seperated training data. The model we
are trying to train will be predicting the highest grade a climber has
ever entered as completed in the database. To try to predict this grade
we will be using the climbers height, weight, year started, and, sex.
The parameters and output make sense. These are all factors that seem to
influence climbing ability. Some of the possible limits of this model
are the large amount of missing data in the input data. There were
thousands of rows where people hadn't entered one of these things.

\begin{Shaded}
\begin{Highlighting}[]
\NormalTok{grade_model <-}\StringTok{ }\KeywordTok{lm}\NormalTok{(train_data, }\DataTypeTok{formula =}\NormalTok{ grade_id }\OperatorTok{~}\StringTok{ }\NormalTok{height }\OperatorTok{+}\StringTok{ }\NormalTok{weight }\OperatorTok{+}\StringTok{ }\NormalTok{started }\OperatorTok{+}\StringTok{ }\NormalTok{sex)}
\KeywordTok{summary}\NormalTok{(grade_model)}
\end{Highlighting}
\end{Shaded}

\begin{verbatim}
## 
## Call:
## lm(formula = grade_id ~ height + weight + started + sex, data = train_data)
## 
## Residuals:
##     Min      1Q  Median      3Q     Max 
## -54.947  -5.276   1.003   6.203  32.732 
## 
## Coefficients:
##               Estimate Std. Error t value Pr(>|t|)    
## (Intercept) 844.398692  12.437896  67.889   <2e-16 ***
## height        0.003208   0.002811   1.141    0.254    
## weight       -0.225630   0.005401 -41.774   <2e-16 ***
## started      -0.386240   0.006197 -62.328   <2e-16 ***
## sex          -7.758402   0.170345 -45.545   <2e-16 ***
## ---
## Signif. codes:  0 '***' 0.001 '**' 0.01 '*' 0.05 '.' 0.1 ' ' 1
## 
## Residual standard error: 8.702 on 34063 degrees of freedom
## Multiple R-squared:  0.1592, Adjusted R-squared:  0.1591 
## F-statistic:  1612 on 4 and 34063 DF,  p-value: < 2.2e-16
\end{verbatim}

This test how accurate our model is using the predicted values. Before
doing this we need to clean the data back to whole numbers since climbs
are based on the numbers and cant have decimals. Then we use the RMSE
functions to calculate our value. lowest test sample RMSE is the
preferred model.

\begin{Shaded}
\begin{Highlighting}[]
\NormalTok{grade_predictions <-}\StringTok{ }\NormalTok{grade_model }\OperatorTok\StringTok{ }\KeywordTok{predict}\NormalTok{(test_data)}
\NormalTok{grade_predictions <-}\StringTok{ }\KeywordTok{as.data.frame}\NormalTok{(grade_predictions)}
\KeywordTok{colnames}\NormalTok{(grade_predictions)[}\KeywordTok{colnames}\NormalTok{(grade_predictions) }\OperatorTok{==}\StringTok{ "grade_predictions"}\NormalTok{] <-}\StringTok{ "grade_id"}
\NormalTok{grade_predictions[}\StringTok{"grade_id"}\NormalTok{] <-}\StringTok{ }\KeywordTok{floor}\NormalTok{(grade_predictions[}\StringTok{"grade_id"}\NormalTok{])}
\KeywordTok{RMSE}\NormalTok{(grade_predictions, test_data}\OperatorTok{$}\NormalTok{grade_id)}\OperatorTok{/}\KeywordTok{mean}\NormalTok{(test_data}\OperatorTok{$}\NormalTok{grade_id)}
\end{Highlighting}
\end{Shaded}

\begin{verbatim}
## [1] 0.1591043
\end{verbatim}

Now we are going to look at another source of data. This data was from
the American Alpine Club in 2014 and shows the number of acciedents
reported in the mountains during activites like climbing and
moutainering. We are going to load this data from a csv file.

\begin{Shaded}
\begin{Highlighting}[]
\NormalTok{injuries =}\StringTok{ }\KeywordTok{read.csv}\NormalTok{(}\StringTok{"climbInjuries.csv"}\NormalTok{, }\DataTypeTok{header =} \OtherTok{TRUE}\NormalTok{)}
\end{Highlighting}
\end{Shaded}

One of the things I am interested in seeing if there is any relaionship
between the number of climbers who started in a year and the number of
accidents. My goal here is see if there is some potential link between
the the number of new climbers and the number of injuries in a given
year. Here we are going to clean the data. First we need to generate
some data by counting the number of users who started in each given
year. Then we are going to remove any years not in our new data set. Our
new data set is alread clean and is ready to be merge with the data we
are generating below before we begin to plot it.

\begin{Shaded}
\begin{Highlighting}[]
\NormalTok{new_per_year <-}\StringTok{ }\KeywordTok{count}\NormalTok{(userInfo, year_started)}
\KeywordTok{colnames}\NormalTok{(new_per_year)[}\KeywordTok{colnames}\NormalTok{(new_per_year) }\OperatorTok{==}\StringTok{ "year_started"}\NormalTok{] <-}\StringTok{ "year"}
\KeywordTok{colnames}\NormalTok{(new_per_year)[}\KeywordTok{colnames}\NormalTok{(new_per_year) }\OperatorTok{==}\StringTok{ "n"}\NormalTok{] <-}\StringTok{ "num_new_climbers"}
\NormalTok{new_per_year <-}\StringTok{ }\NormalTok{new_per_year[}\OperatorTok{!}\KeywordTok{rowSums}\NormalTok{(new_per_year[}\DecValTok{1}\NormalTok{] }\OperatorTok{<}\DecValTok{1981}\NormalTok{),]}
\NormalTok{new_per_year <-}\StringTok{ }\NormalTok{new_per_year[}\OperatorTok{!}\KeywordTok{rowSums}\NormalTok{(new_per_year[}\DecValTok{1}\NormalTok{] }\OperatorTok{>}\StringTok{ }\DecValTok{2013}\NormalTok{),]}
\NormalTok{new_per_year <-}\StringTok{ }\KeywordTok{na.omit}\NormalTok{(new_per_year)}
\NormalTok{injuries <-}\StringTok{ }\KeywordTok{merge}\NormalTok{(injuries,new_per_year, }\DataTypeTok{by=}\StringTok{"year"}\NormalTok{)}
\end{Highlighting}
\end{Shaded}

showing injuries table for documentation purpose

\begin{Shaded}
\begin{Highlighting}[]
\KeywordTok{colnames}\NormalTok{(injuries)}
\end{Highlighting}
\end{Shaded}

\begin{verbatim}
## [1] "year"             "num_of_accidents" "num_new_climbers"
\end{verbatim}

Finally we are going to graph the number of new climbers per year and
the number of injuries per year to see if there is a relationship.

\begin{Shaded}
\begin{Highlighting}[]
\NormalTok{injury_plot <-}\StringTok{ }\KeywordTok{ggplot}\NormalTok{(injuries, }\KeywordTok{aes}\NormalTok{(}\DataTypeTok{x=}\NormalTok{year)) }\OperatorTok{+}\StringTok{ }\KeywordTok{geom_point}\NormalTok{(}\KeywordTok{aes}\NormalTok{(}\DataTypeTok{y=}\NormalTok{num_new_climbers, }\DataTypeTok{colour =}\StringTok{'blue'}\NormalTok{)) }\OperatorTok{+}
\StringTok{    }\KeywordTok{geom_point}\NormalTok{(}\KeywordTok{aes}\NormalTok{(}\DataTypeTok{y=}\NormalTok{num_of_accidents, }\DataTypeTok{colour =}\StringTok{'red'}\NormalTok{)) }\OperatorTok{+}\StringTok{ }\KeywordTok{ylab}\NormalTok{(}\StringTok{"Number of Climbers"}\NormalTok{) }\OperatorTok{+}\StringTok{ }\KeywordTok{xlab}\NormalTok{(}\StringTok{"Year"}\NormalTok{) }\OperatorTok{+}\StringTok{    }\KeywordTok{scale_color_manual}\NormalTok{(}\DataTypeTok{labels =} \KeywordTok{c}\NormalTok{(}\StringTok{"# of New Climbers"}\NormalTok{, }\StringTok{"# of Accidents"}\NormalTok{), }\DataTypeTok{values =} \KeywordTok{c}\NormalTok{(}\StringTok{"blue"}\NormalTok{, }\StringTok{"red"}\NormalTok{))}

\NormalTok{injury_plot}
\end{Highlighting}
\end{Shaded}

\includegraphics{part2_files/figure-latex/unnamed-chunk-10-1.pdf}


\end{document}
